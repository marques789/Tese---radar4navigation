\chapter{Conclusion and Future Work}


\section{Conclusion}
In this work we it was examined in detail how the \ac{FMCW} \ac{radar} and \ac{LiDAR} technology work and how they are used for certain applications. After this, it was implemented a robotic navigation platform using \ac{ROS}, \ac{TI} \ac{mmWave} devices and the 2D-\ac{LiDAR} that is ready to use for indoor navigation tasks. After that was implemented we devised various experiments that examined the use of the previous sensors for obstacle detection. It was concluded that the \ac{FMCW} \ac{radar} perceived certain objects which the \ac{LiDAR} can not. We also concluded that the robot behaved correctly in unstructured environments only using the \ac{FMCW} \ac{radar} as an obstacle detector. This means the robotic platform may prove to be an alternative or support sensor source to the \ac{LiDAR} when it comes to indoor spaces. Finnaly we also designed a new map layer that uses the relative radial velocity provided by the \ac{radar} in order to produce safer trajectories for the mobile robot. We conclude that the use of the \ac{FMCW} \ac{radar} is a low cost sensor that provide a bundle of information that can be used for optimizing the navigation of mobile robots.
\section{Future Work}
Taking into account the work done in this dissertation there are multiple objectives that we can try to perform in the near future. Firstly, in the first experiment we only experiment with a small amount of different obstacles. The number of objects can be expanded to try and find more where the 2-D\ac{LiDAR} fails relatively to the \ac{FMCW} \ac{radar}. Another objective to take in mind is the further improvement of the plugin costmap layer proposed in this work, the parameterization and optimization may be key factor for the successful use of this layer to combat unstructured dynamic environments. Lastly in this work we only tackled the obstacle avoidance capabilities of the \ac{FMCW} \ac{radar}, in the future we can try to use it for mapping or even localization purposes.
%\begin{itemize}
 %   \item Evaluation of the performance of the \ac{FMCW} \ac{radar} for different objects that were not performed in this work.
%    \item Further development and optimization of the implemented doppler layer.
%    \item Construction of an independent navigation system without requirement of the \ac{ROS} navigation stack
 %   \item Clustering of obstacles for better perception of the robot.
 %   \item Development of new costmap layers to manipulate the master costmap taking into account the application we are trying to build.
  %  \item Utilizing the \ac{FMCW} \ac{radar} for localization purposes.
 %   \item Utilizing the \ac{FMCW} \ac{radar} for mapping  purposes.
%\end{itemize}

