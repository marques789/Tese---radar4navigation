%%%%%%%%%%%%%%%%%%%%%%%%%%%%%%%%%%%%%%%%%%%%%%%%%%%%%%%%%%%%%%%%%%%%%%%%%%%%%%%%%%%%%%%%%%%%%%%%%%%
\chapter{Introduction} \label{ch:introduction}

%This chapter describes the motivations for this dissertation, followed by its objectives. The contributions done during this dissertation are also presented, along with a succinct description of the documents structure.

%The \ac{VANET} that was used present in his architecture different entities: an \ac{LMA} which is responsible for managing the \ac{VANET}, and also to store information about the overall network status; and the \ac{RSU}s and \ac{OBU}s are also real systems that are called NetRiders - single board computers containing \ac{WAVE}, \ac{WiFi}, \ac{GPS}, and cellular interfaces. This \ac{VANET} is capable of \ac{MH} where it sends the traffic through the different technologies available resulting in a better load balancing, this load balancing mechanism feature is integrated at the \ac{LMA} which is responsible to optimize the traffic division having in mind the different technologies and throughput available for each \ac{PoA}. This \ac{VANET} is also capable of performing \ac{NC} to provide loss reduction, where the encoding is done at the \ac{RSU}s level, where the packets will be encoded through the wireless communications between the \ac{RSU}s and the \ac{OBU}s. 



%%%%%%%%%%%%%%%%%%%%%%%%%%%%%%%%%%%%%%%%%%%%%%%%%%%%%%%%%%%%%%%%%%%%%%%%%%%%%%%%%%%%%%%%%%%%%%%%%%%
%\section{Motivation} 




\section{Context}
Automation of mobile robots has been a major subject of interest for a long time. This lead to the development of various algorithms to optimize the behavior of robotic platforms when doing navigation tasks. However the perception of this robots is usually done by some type of 

\section{Objectives}
The main objective of this project is to evaluate the performance of \ac{FMCW} radar technology for the use in robot autonomous indoor navigation. To do this we must first convert the radar data into the \ac{ROS} platform and process it in order to create suitable information for the robot to perceive its environment. Once that is completed we can feed it to our navigation system and study how well the robot performs indoor navigation tasks.
It will be compared the performance of this sensors with the more commonly used and more expensive \ac{LIDAR} technology.

\section{Contributions}
The main contribution to this dissertation is the study of \ac{FMCW} radars as a supportive device for indoor navigation.
\section{Document Structure}
The dissertation is arranged as follows:

Chapter 2 will discuss a brief history in the development of autonomous mobile robots, what sensors they use, and the evolution of the \ac{FMCW} radars over the years.

Chapter 3 will provide a brief overview of some basic concepts regarding this dissertation's work regarding navigation concepts, sensors working principles and the ROS framework platform.


Chapter 4 we will explain the inner workings of the ROS Navigation Stack software and how it can be used to obtain autonomous navigation in most robots as well as how to tune its parameters to achieve the desired navigation behavior.




