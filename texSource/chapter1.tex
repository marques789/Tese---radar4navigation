%%%%%%%%%%%%%%%%%%%%%%%%%%%%%%%%%%%%%%%%%%%%%%%%%%%%%%%%%%%%%%%%%%%%%%%%%%%%%%%%%%%%%%%%%%%%%%%%%%%
\chapter{Introduction} \label{ch:introduction}

%This chapter describes the motivations for this dissertation, followed by its objectives. The contributions done during this dissertation are also presented, along with a succinct description of the documents structure.

%The \ac{VANET} that was used present in his architecture different entities: an \ac{LMA} which is responsible for managing the \ac{VANET}, and also to store information about the overall network status; and the \ac{RSU}s and \ac{OBU}s are also real systems that are called NetRiders - single board computers containing \ac{WAVE}, \ac{WiFi}, \ac{GPS}, and cellular interfaces. This \ac{VANET} is capable of \ac{MH} where it sends the traffic through the different technologies available resulting in a better load balancing, this load balancing mechanism feature is integrated at the \ac{LMA} which is responsible to optimize the traffic division having in mind the different technologies and throughput available for each \ac{PoA}. This \ac{VANET} is also capable of performing \ac{NC} to provide loss reduction, where the encoding is done at the \ac{RSU}s level, where the packets will be encoded through the wireless communications between the \ac{RSU}s and the \ac{OBU}s. 



%%%%%%%%%%%%%%%%%%%%%%%%%%%%%%%%%%%%%%%%%%%%%%%%%%%%%%%%%%%%%%%%%%%%%%%%%%%%%%%%%%%%%%%%%%%%%%%%%%%
%\section{Motivation} 




\section{Context}
Automation of mobile robots has been a major subject of interest for a long time. The improvement of navigation tasks. This lead to the development of various algorithms to optimize the behavior of robotic platforms when doing navigation tasks. 

\section{Objectives}
The main objective of this of this project is to evaluate the performance of FMCW radar technology for the use in robot autonomous indoor navigation. It will be compared the performance of this sensors with the more commonly used and more expensive lidar technology.


\section{Turtlebot2}


\section{Contributions}



\section{Document Structure}

