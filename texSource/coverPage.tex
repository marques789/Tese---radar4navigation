%
% Cover page (use only one of the first two \TitlePage)
%

% First alternative, with a figure
\TitlePage
  %\GRID  % for debugging ONLY
  \HEADER{\BAR\FIG{\includegraphics[height=60mm]{imgs/uaLogoNew}}} % the \FIG{} is optional
         {\ThesisYear}
  \TITLE{Francisco \newline Marques dos \newline Santos}
        {Study of the use of radar for indoor navigation \newline Estudo do uso
do radar na navegação em espaços interiores}
        
\EndTitlePage
\titlepage\ \endtitlepage % empty page


%
% Initial thesis pages
%

\TitlePage
  \HEADER{}{\ThesisYear}
  \TITLE{Francisco \newline Marques dos \newline Santos}
        {Indoor navigation using the TurtleBot platform \newline Navegação em ambientes interiores utilizando a plataforma TurtleBot}
  \vspace*{15mm}
  \TEXT{}
       {Dissertação apresentada à Universidade de Aveiro para cumprimento dos
requisitos necessários à obtenção do grau de Mestre em Engenharia Electrónica e Telecomunicações, realizada sob a orientação científica do Professor Doutor Artur Pereira, Professor Catedrático do Departamento de
Electrónica e Informática da Universidade de Aveiro e co-orientação científica do Doutor Eurico Pedrosa, Investigador da
Universidade de Aveiro.
        }
        \vspace*{85mm}
  %\TEXT{}
       %{"God Placed the Best Things in Life on the Other Side of Fear." \newline - Will Smith}
\EndTitlePage
\titlepage\ \endtitlepage % empty page

\TitlePage
  \vspace*{55mm}
  \TEXT{\textbf{o j\'uri~/~the jury\newline}}
       {}
  \TEXT{presidente~/~president}
       {\textbf{Professor Doutor Ant\'onio Lu\'is Jesus Teixeira}\newline {\small
        Professor Associado c/ Agrega\c c\~ao do 
        Departamento de Electr\'onica, Telecomunica\c c\~oes e Inform\'atica da Universidade de Aveiro}}
  \vspace*{5mm}
  \TEXT{vogais~/~examiners committee}
       {\textbf{Professor Doutor Rui Pedro de Magalh\~aes Claro Prior}\newline {\small
        Professor Auxiliar do 
        Departamento de Ci\^encia de Computadores da Faculdade de Ci\^encias da Universidade do Porto}}
  \vspace*{5mm}
  \TEXT{}
       {\textbf{Professor Doutor Andr\'e Ventura da Cruz Marnoto Z\'uquete}\newline {\small
        Professor Auxiliar do Departamento de Electr\'onica, Telecomunica\c c\~oes e Inform\'atica da Universidade de Aveiro (Orientador)}}
\EndTitlePage
\titlepage\ \endtitlepage % empty page

\TitlePage
  \vspace*{55mm}
  \TEXT{\textbf{agradecimentos}}
       {Em primeiro lugar queria agradecer à minha mãe por todo o apoio ao
        longo do meu percurso académico. Ao professor Artur Pereira pela paciência e constante entusiasmo que me deu a prevalência de continuar e acabar o projeto.
     Ao meu co-orientador, Eurico Pedrosa, pela disponibilidade, espírito
crítico e apoio constante ao longo do desenvolvimento deste trabalho.
Finalmente, agradeço ao grupo de investigação RETIOT pela ajuda
crucial nos testes práticos realizados ao longo desta dissertação.
Muito obrigado a todos.
       }
  \TEXT{}
       {%Gostaria também de agradecer aos meus colegas que tive a sorte de conhecer durante o meu percurso acadêmico nestes últimos anos. Amizades que se criaram que certamente se levarão para o resto da vida.
       }

  \TEXT{}
       {%Agradeço também ao nosso grupo de investigação por todo o conhecimento, entreajuda e apoio que demonstraram ao longo deste trabalho desenvolvido. Gostaria de fazer um agradecimento especial ao João Pereira, João Ribeiro, João Rocha e ao Rui Lopes pela disponibilidade e ajuda demonstrada e por todo o ambiente de trabalho que foi possível ter junto deles. 
       }
  \TEXT{}
       {%Gostaria também de agradecer ao professor André Zúquete e ao Dr. Miguel Luís pela orientação, pelo espírito crítico e pela ajuda prestada durante o desenvolvimento desta dissertação. Um especial agradecimento também à professora Susana Sargento por me ter dado a oportunidade de integrar este ambicioso grupo de investigação, por me ter proporcionado a possibilidade de trabalhar fora da minha zona de conforto contribuindo assim muito para o meu conhecimento científico e que sempre esteve presente durante todo o percurso deste trabalho desenvolvido, tendo sem dúvida uma importância muito grande naquilo que é este trabalho e o que representa para mim.
       }
   \TEXT{}     
       {%Por fim, agradeço à Fundação para a Ciência e Tecnologia pelo suporte financeiro através de fundos nacionais no âmbito do projeto, S2MovingCity CMUP-ERI/TIC/0010/2014.
        }
\EndTitlePage
\titlepage\ \endtitlepage % empty page

\TitlePage
  \vspace*{55mm}
  \TEXT {\textbf{Resumo}}
       {Nos últimos anos, o uso de robôs móveis autónomos que operam em ambientes indoor aumentou significativamente. Eles são usados não apenas
        em contexto de indústria, mas estão surgir nas nossas casas e ambientes de escritório. Eles são tipicamente usados para transporte, telepresença e limpeza. A percepção do robô geralmente depende de medidor de distâncias a laser, sensores de profundidade de infravermelho ou a típica câmera. No entanto, os recentes avanços na tecnologia de radar de onda contínua modulada em frequência permitem uma nova solução para o problema de percepção. Neste trabalho, descobriremos como  as tecnologias \ac{LiDAR} e o \ac{FMCW} \ac{radar} podem ser usados para perceber o ambiente em redor do robô dele e como elas podem permitir  navegação robótica autónoma em ambientes de interior. O principal objetivo desta dissertação é
        entender como estas tecnologias funcionam e se o uso do radar FMCW para
        percepção pode ser vantajosa para tarefas de navegação.
       }
       \TEXT{}
       { Primeiro, será apresentado um estudo comparativo entre cada tecnologia, incluindo o princípio operacional de cada uma, o trabalho em que estão ser usados e quais as suas limitações que cada uma tem. Depois descobriremos como podemos usar o Robot Operating
        System (ROS) e algoritmos de última geração para combater o problema de navegação autónoma. Após a discussão anterior, descreveremos
        os componentes de hardware e software e como eles estão interconectados
        produzir uma plataforma robótica adequada que será usada para executar tarefas de navegação.
        Com a plataforma de navegação criada, propomos várias experiências que
        tentam avaliar o desempenho do LiDAR-2D e o radar FMCW e  como
        detectores de obstáculos e verificar vantagens criadas usando o
        último. Finalmente, será realizado um trabalho exploratório que tenta usar o
        leituras de canal doppler do radar para ajudar em trajetórias de percurso mais seguras para ambientes indoor sociais.}
       \TEXT{}     
       {%Ambos os mecanismos foram avaliados em cenários de laboratório com sistemas reais. Os resultados obtidos relativos ao envio das mensagens de controlo mostram que esta nova abordagem é capaz de fornecer uma comunicação com maior fiabilidade, reduzindo as perdas de pacotes no caso de uma desconexão abrupta, e quando na presença de outras tecnologias e ligações. Quanto à proposta de muti-tecnologia para o Network Coding, os testes experimentais avaliaram o seu impacto na taxa de entrega de pacotes efetiva e no atraso de transmissão. Os resultados comparativos evidenciam que, apesar de ter um pequeno impacto no atraso dos pacotes em comparação com a abordagem que considera o Network Coding em cada tecnologia de forma independente, a abordagem de multi-tecnologia apresenta uma melhor taxa de entrega.
       }
 
       
\EndTitlePage
\titlepage\ \endtitlepage % empty page

\TitlePage
  \vspace*{55mm}
  \TEXT{\textbf{Abstract}}
       {Over the past few years the use of autonomous mobile robots that operate in indoor environments has grown significantly. They are used not only in industry settings but are creeping in on our homes and office environments. They are commonly used for application such as transportation, telepresence and cleaning. The robot's perception is usually dependent of laser range finders, infra red depth sensors or the typical camera. However the recent advancements on frequency modulated continuous wave radar technology permits a new solution for the perception problem. In this work we will uncover how \ac{LiDAR} and \ac{FMCW} \ac{radar} are used to perceive the surrounding robot environment and how they enable robotic indoor navigation. The main purpose of this dissertation is to understand how this technologies work and if the use of the \ac{FMCW} \ac{radar} perception can be advantageous for indoor navigation tasks. 
       }
       \TEXT{}     
       {First, a comparative study between each  technology will be presented including the operating principle of each one, applications and what limitations they have. After that we will uncover how we can use the \ac{ROS} framework and state of the art algorithms  to address the autonomous navigation problem. After the previous discussion we will describe the robot's hardware and software components and how they are interconnected to produce a suitable robotic platform that will be used to do navigation tasks.
       %his dissertation enhances the communication quality of a \ac{MH} vehicular network by improving its mobility protocol and the \ac{NC} mechanisms. Specifically, changes were performed to ensure the reliability of control mobility messages to help the infrastructure to react faster to the wireless communication conditions of a mobile node.
       %from the mobility protocol developed in our research group allowing a more reliable implementation reducing the packet losses by making use of the best connection possible to send the disconnect message, alongside this, some minor aspects were improved at the mobility protocol. Another objective represent a major improvement on the Network coding implementation that is currently used in our \ac{VANET}, were the encoding process is performed at the start of our network, making use of all the available technologies to reach the end-user, resulting in a multi-technology approach for network coding. 
       %On a different perspective, it has been provided a mechanism to enable \ac{NC} through different technologies being used in \ac{MH}, and making use of all technologies simultaneously to code and recover the packets. %this improvement should also improve the available througput when \ac{NC} is applied.
       %Another proposed architecture in this work is the integration of a multi-technology approach for the network coding directly with the mobility protocol resulting in an major improvement for the available throughput.
        }
        \TEXT{}     
       {With the navigation platform created we propose various experiments that evaluates the performance of the   2-D\ac{LiDAR} and the \ac{FMCW} radar as obstacle detectors and explore the advantages that are presented by using the latter. Finally an exploratory work will be conducted that tries to use the doppler channel readings of the \ac{radar} to aid in more safe pathing trajectories for social indoor environments.
       %The tests executed in this dissertation to evaluate the improvement and stability impact on the mobility protocol, originated by the new disconnect mechanism proposed for the mobility protocol, were performed with some laboratory scenarios.
       %Both approaches were evaluated with real systems in a laboratory scenario. 
       %The comparative results with the previous implementation shows a direct improvement on the reliability reducing the packet losses presented in case of an abrupt disconnection when in presence of other available connections. For the multi-technology architecture proposed for the network coding, some tests using different coding configurations (generation and buffer sizes) were performed, and they were evaluated by analyzing its impact on the packet loss recovery ratio and the delay. The comparative results show that the multi-technology approach has a better delivery ratio when compared to the single-technology, despite the small impact on the packet delay, and the multi-technology approach integrated with the mobility protocol show that it is possible to achieve a better throughput without a negative trade-off on the delivery ratio or the presented delay.
       %The obtained results on the reliability of the control messages show that the new approach is able to provide higher communication reliability, reducing the packet losses presented in case of an abrupt disconnection, and when in presence of other connections. For the multi-technology architecture proposed for the \ac{NC}, the experimental tests evaluated its impact on the effective delivery ratio and the delay. The comparative results show that the multi-technology approach integrated with \ac{MH} has a better delivery ratio when compared to the single-technology, despite the small impact on the packet delay.
        }
       % \TEXT{}     
       %{The tests for the multi-technology architecture proposed for the network coding will be evaluated using different %coding configurations (generation and buffer sizes) and by analyzing its impact on the packet loss recovery ratio. %The comparative results show that the multi-technology approach has a better delivery ratio when compared to the %single-technology, despite the small impact on the packet delay, and the multi-technology approach integrated with %the mobility protocol show that it is possible to achieve a better throughput without a negative trade-off on the %delivery ratio or the presented delay.
    %    }
\EndTitlePage
\titlepage\ \endtitlepage % empty page