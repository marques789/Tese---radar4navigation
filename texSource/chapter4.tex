\chapter{Robotic Navigation Platform Descripton}

In this section we will describe the principal components in terms of hardware and software and how they are interconnected to create a suitable robotic platform that will later be used for comparing the performance of \ac{LiDAR} and \ac{FMCW} \ac{radar} as obstacle avoidance sensors.


\section{Hardware}
The basic hardware for this work will be a  modified version of the  turtlebot2 platform connected to an \ac{FMCW} \ac{radar} and a 2D-\ac{LiDAR} as proximity sensors.


\subsection{Turtlebot2}
TurtleBot 2 (Fig. \ref{fig:t2}) is one of the  most popular low cost personal robots around. It is completely run by open source software which makes it exceptional for research and educational purposes. The robot has been developed by the Korean company Yujin Robotics in collaboration with Willow Garage, its differential kinematics mobile platform can be used for multiple applications, due to the huge number of available ROS packages. It comes with a kobuki base a processing unit Intel NUC, the 2-D \ac{LiDAR} Hokuyo URG-04LX-UG01 Scanning Laser Rangefinder and finally a Kinect for XBOX 360.

\begin{figure}[h] 
\centerline{\includegraphics [width=0.3 \textwidth]{imgs/chapter4/turtlebot2.png}}
\caption{Turtlebot 2 platform}
\label{fig:t2}
\end{figure}

\subsubsection*{Technical Specifications}
When it comes to specs the robots dimension is 354 x 354 x 420 mm as shown in Fig. \ref{fig::t2specs}, its weight is 6.3 Kg with a max payload of 5 Kg which means it is able to attach lots of devices with relatively heavy if needed. The maximum translational  speed of it is 0.7 m/s and the maximum rotational speed is 180º/s . It as a gyroscope with 1 axis(110º/s), odometer at 52 ticks/encoder and  bumpers on left right and center among other components. 

\begin{figure}[h] 
\centerline{\includegraphics [width=1.0 \textwidth]{imgs/chapter4/tspecs.png}}
\caption{Turtlebot 2 dimension specifications}
\label{fig::t2specs}
\end{figure}


\subsection{FMCW radar}
The radar board chosen for this work is the mmwave Texas Instruments AWR1642BOOST (Fig.\ref{fig:awr}). This is a recently distributed \ac{FMCW} radar dedicated for short range applications. It is an easy-to-use evaluation board for the AWR1642 automotive radar sensor which is connected to the micro-controller unit (MCU) LaunchPad. To develop software it has  on-chip C67x DSP core and low-power ARM Cortex-R4F controllers which include onboard emulation for programming and debugging.
It requires a 5V > 2.5 A supply brick with a 2.1-mm barrel jack to run.
The device supports a wide RF bandwidth of 77-81 GHz that permits good range, velocity  and angle resolution.

\begin{figure}[h] 
\centerline{\includegraphics [width=0.5 \textwidth]{imgs/chapter4/awr1642.jpg}}
\caption{Texas Instruments AWR1642BOOST evaluation board}
\label{fig:awr}
\end{figure}

\subsection{LiDAR}
Besides the \ac{FMCW} \ac{radar} the Hokuyo URG-04LX-UG01 Scanning Laser Rangefinder (Fig. \ref{fig:lidar}) is also attached to the platform.

\begin{figure}[h] 
\centerline{\includegraphics [width=0.5 \textwidth]{imgs/chapter4/lidar.jpg}}
\caption{Hokuyo URG-04LX-UG01 Scanning Laser Rangefinder}
\label{fig:lidar}
\end{figure}

This sensor is an inexpensive 2D-\ac{LiDAR} that is based on phase difference measurement. It retrieves information of the surrounding environment by scanning an area of 240º with 0.36º angular resolution. The maximum range of it is about 4 meters and its range resolution is 1mm. Its scan update time is 100ms/scan and its weight is 360 g.  As for the power supply it only needs a 5V DC provided by the USB connection as shown in figure \ref{fig:lidarS}.



\begin{figure}[h] 
\centerline{\includegraphics [width=0.6 \textwidth]{imgs/chapter4/lidarS.png}}
\caption{Hokuyo URG-04LX-UG01 detection diagram}
\label{fig:lidarS}
\end{figure}

\subsection{Robotic Platform}
The out of the box kobuki platform was modified in order to include a processing unit, a 2D scanning \ac{LiDAR} and a \ac{FMCW} radar as proximity sensors. The modified version is displayed in Fig.\ref{fig::turlebot2M}. 

\begin{figure}[h] 
\centerline{\includegraphics [width=0.5 \textwidth]{imgs/chapter4/turtlebot2.PNG}}
\caption{Modified Turtlebot2 used in this work}
\label{fig::turlebot2M}
\end{figure}


\section{Software}

\subsection{\ac{PCL}}
%Describe PCL

Since we are dealing with point clouds coming from the \ac{FMCW} \ac{radar} and the 2-D\ac{LiDAR} than we need some type of ways to handle and manipulate them. For that the open source libraries called \ac{PCL} is the more indicated place to process and manipulate this type of information. A pointcloud is a collection of multi-dimensional points and is commonly used to represent three-dimensional data \cite{pcl}. This points are often just designed to locate points in x,y,z but more dimensions can be added as we will see later on, for the FMCW radar it has 6 dimensions. \ac{PCL} provides open source state of the art library modules that enables  filtering, feature estimation, surface reconstruction, registration, model fitting and segmentation. In our case we will mainly use its passthrough filters. 



\subsection{ROS radar driver}
Texas Instruments provides a ros package that interfaces radar data to the ROS message format \cite{tisetup}. The incoming radar TLV data from the mmWave EVM is decoded in order to create a  \textbf{PointCloud2} type ROS message.
This point cloud follows the detected objects frame described in the mmWave demo data structure \cite{mmdata} represented in Fig. \ref{fig:demodata}.
\begin{figure}[!htb]
    \centering
    \includegraphics[scale=0.8]{imgs/chapter4/demodata.png}
    \caption{Part of the mmwave demo packet containing object detection information. This fields will be used to construct the ROS PointCloud2 \cite{mmdata}.}
    \label{fig:demodata}
\end{figure}
Each point has 6 fields:
\begin{itemize}
\item \textbf{x (m)} - position x of the detected  object in the frame of the radar.
\item \textbf{y (m)} - position y.
\item \textbf{z (m)} - position z (for 2D devices this is equal to zero).
\item \textbf{range (m)} - range of the object relative to the radar frame.
\item \textbf{doppler (m/s)} - radial velocity of the object relative to the radar frame.
\item \textbf{intensity} - relative power of the received signal corresponding to that object.
\end{itemize}
\subsubsection{Running the driver and observing the outputted pointcloud in rviz}
After installing the driver \cite{tisetup} to start receiving and visualizing the incoming radar data run the following commands while in your ROS workspace directory:
 \begin{lstlisting}[language=bash]
source devel/setup.bash
roslaunch ti_mmwave_rospkg rviz_1642_2d.launch
\end{lstlisting}
This will launch the TI mmWave ROS Driver nodes that configure the EVM and publish the radar data. After that rviz will load and you should be able to visualize the objects detected by the radar. Figure \ref{fig:radar_rviz} shows an example of the radar data displayed on rviz. The points correspond to objects   detected by the radar.
\begin{figure}[!htb]
    \centering
    \includegraphics[scale=0.2]{imgs/chapter4/radardatarviz.png}
    \caption{Pointcloud2 radar data displayed by rviz}
    \label{fig:radar_rviz}
\end{figure}

\subsection{Visualization of the radar point cloud}
Plotting the points in the XYZ space is not enough to fully visualize the radar data sent since each point also gives velocity and intensity information.

We can visualize it by using markers such as arrows or text in rviz.
Figure \ref{fig:doppler_marker} displays the radial velocity of each point with an arrow. The size of the arrow indicates how fast the object is going. Figure \ref{fig:intensity_marker} shows the intensity values of each object in text. This type of visualization will be useful later on when we try to filter the cloud.

\begin{figure}[!htb]
    \centering
    \includegraphics[scale=0.25]{imgs/chapter4/dopplermarker.png}
    \caption{Arrow markers displaying the points radial velocity}
    \label{fig:doppler_marker}
\end{figure}

\begin{figure}[!htb]
    \centering
    \includegraphics[scale=0.25]{imgs/chapter4/intensitymarker.png}
    \caption{Text markers displaying the points intensity values}
    \label{fig:intensity_marker}
\end{figure}
\subsection{Chirp profile configuration file}
The characteristics of this point cloud such as publishing rate, range resolution, maximum range, velocity resolution, maximum velocity depends on the chirp profile configuration file loaded in the EVM. This file is located in the \textbf{"cfg"} folder of the ti mmwave package and can be replaced in order to accommodate a given application.

The easiest way to create a chirp configuration file is the mmWave Demo Visualizer. With it you can auto generate a configuration file given a set of specifications.
Another way of doing this is manually. This however requires the understanding of the radar operating principle and the configuration commands.

\section{Radar Data Processing}
Now that we have the radar constantly giving us a point cloud we can further process to retrieve the information we want. 
In our case we will make use of \textbf{passthrough filters} to remove unwanted points and \textbf{euclidean clustering} to identify groups of points that belong to the same object.

This operations will be done by using the \textbf{Point Cloud Library}. PCL offers a wide variety of modules that handle point clouds but in our case we will only use it for the previously mentioned operations.

\begin{figure}[h] 
    \begin{minipage}[b]{.49\linewidth}
        \includegraphics[height=5cm,width=\linewidth]{imgs/chapter4/filt.png}
        \subcaption{Non filtered pointcloud}
        \label{fig:fft}
    \end{minipage}
    \begin{minipage}[b]{.49\linewidth}
        \includegraphics[height=5cm,width=\linewidth]{imgs/chapter4/notfilt.png}
        \subcaption{Filtered pointcloud by intensity]}
        \label{fig:fft2}
    \end{minipage}
    \caption{Example of filtering the pointcloud}
    \label{fig:distance determination}
\end{figure}


\section{Software setup for ROS navigation}
The software used in this project is divided in three stages. First one is the interface between the radar board TLV data and the ROS point cloud message format. Then using \ac{PCL} a filtering operation is added to remove false detections (this will be explained later). Finally, the filtered radar data as well as the \ac{LiDAR} data are fed to the ROS Navigation Stack as sensor sources. 
%%Block diagram

Figure \ref{fig::softsetup} shows a simple block diagram describing  the software setup.
\begin{figure}[h] 
\centerline{\includegraphics [width=0.7 \textwidth]{imgs/chapter4/bd.png}}
\caption{Software blockdiagram}
\label{fig::softsetup}
\end{figure}





\section{Summary}
In this chapter it was explained what tools were used for properly setting up the turtlebot robot for indoor navigation.

